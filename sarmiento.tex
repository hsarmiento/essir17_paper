% ewic.tex for classfile V2.04, 6 July 2011

\documentclass{ewic}
%\documentclass[cm]{ewic}
\usepackage{graphicx}  %Required
\begin{document}

\runningheads{Sarmiento}{Detecting Emergency Situations by Inferring Locations in Twitter}

\conference{11th European Summer School in Information Retrieval (ESSIR 2017)}

\title{Detecting Emergency Situations by Inferring Locations in Twitter}

\authorone{Hernan Sarmiento\\
Deparment of Computer Science\\
University of Chile, Beauchef 851, Santiago, 8370456, Chile \\
\email{hsarmien@dcc.uchile.cl}}




\begin{abstract}
Most methods to detect emergency situations using Twitter rely on keywords. The problem with keyword-based methods is the need for training in specific domains for different types of events, for example: earthquakes, typhoons, terrorist attacks, tornadeos, etc.
In contrast, our proposal uses the recurring mention of a country-locations in microblogging messages to identify such events without using keywords and characterize through of inter-arrival times the urgency of situation.
\end{abstract}

\keywords{Information Retrieval, Emergency Situation, Citizen Sensors, Social Media}

\maketitle

\section{Introduction}
Social media has become a major channel for communication during high-impact real events, for example: elections, sports events, emergency situations, etc. In any event, users act as social sensors where they share and post their mood, opinions, photos, videos and exact location by GPS in only eight per cent of messages.

Microblogging plays a critical role during emergency situations because traditional media have damage to infrastructure and affected people can not express their current status. For this reason, researchers have studied the behaviour during these events for to detect, summarize and classify messages with the goal of helping authorities and the general public with situational awareness.

In current works, \cite{ashktorab2014tweedr}, \cite{imranaidr2014}, \cite{kumar2011tweettracker} detect, summarize and classify messages using method rely on keywords over Twitter public streaming API. The problem with these keyword-based methods is the need for training in specific domains for different types of events. In general terms, to label data for classification is hard work because it requires time, human supervision and external sources as crowdsourcing. Furthermore, \cite{olteanu2014} generate a set of keywords based on different disaster datasets, but sometimes specific terms spontaneously arise for one event, for example \textit{\#eqnz} for Earthquake in New Zealand or \textit{\#pabloph} for Typhoon Pablo in Philippines.

We propose a method based on recurring mentions of country-locations in messages' metadata for detecting a new emergency situation without using a set of keywords related to crisis situation. These mentions can occur in the text message, the GPS coordinates, the location of the user profile or a combination of these features.


\section{Methods}
We collect random messages without using keywords or bounding box for three days - before, during and after event - for Mw 6.9 chilean Earthquake (24 April 2017 21:38:28 UTC) using Twitter public streaming API. Addionality, we extract locations for Chile from GeoNames database with at least 5,000 people per location. We create four signals associated with country-locations in the Tweet's metadata. To do this, we inspect the text of the message, the GPS coordinates and the location of the user profile, seeking any mention of country-locations into text (See Table \ref{signalsTweets} for number of tweets by signal):

\begin{itemize}
	\item \textbf{Countrytxt}: text tweet contains a location associated with Chile.
	\item \textbf{Countryusr}: user who shares a message has profile with location associated with Chile.
	\item \textbf{Countrygeo}: tweet contains GPS coordinates in Chile.
	\item \textbf{Countrytxt-usr}: user shares a message that contains a location associated with Chile and his profile as well.
\end{itemize}

\begin{table}[]
	\centering
	\caption{Number of Tweets by Signal}
	\begin{tabular}{|l|l|}
		\hline
		Total Tweet    & 13,655,428 \\ \hline
		Countrytxt     & 66,996     \\ \hline
		Countryusr     & 47,818     \\ \hline
		Countrygeo     & 1,161      \\ \hline
		Countrytxt-usr & 3,519      \\ \hline
	\end{tabular}
	
	\label{signalsTweets}
\end{table}

For each signal, we consider original tweets and retweets, compute their frequency per minute and normalize with respect to the maximum value of each (Figure \ref{signalsTweets}).

\begin{figure}[h]
	
	\centering
	\includegraphics[width=0.5\textwidth]{img/freq_per_minute.png}
	\caption{Normalized Frequency per Minute for Mw 6.9 Chilean Earthquake. (a) \textit{Countrygeo} signal. (b) \textit{Countrytxt} signal.(c) \textit{Countryusr} signal. (d) \textit{Countrytxt-usr} signal.}
\end{figure}

\section{Results and Discussion}

We compare signals when the earthquake occurs. In all cases, signals detect a new emergency situation except the \textit{countrygeo} signal having lower frequency, but represent the exact place where from the message was sent. This result is due to small portion of users using GPS coordinates when sharing a message (about to eight per cent or less). The other signals correctly detect  a new emergency situation because its maximum values coincide with an earthquake's datetime.

However, the signals that represent country-locations in the text message or the location of the user profile, exhibit noise that can generate a burst for other event types. Sometimes these signals have a relative behaviour that depends on the average daily user's activity of country. 

For reducing this noise exhibited by independent features, we combine the \textit{countrytxt} and the \textit{countryusr} signals for generating the \textit{countrytxt-usr} signal. This means that a user with an inferred locality of Chile shares a message that contains an inferred locality of Chile into the text tweet. Moreover, we can consider that a user (probably in Chile) shares information of Chile in his message, thus, the user cares about things that happen in Chile.
\subsection{Characterization of the Signals}

We characterize an emergency situation by using \textit{inter-arrival times} between consecutive social media messages within a sub-time series. The inter-arrival time is defined as $d_{i} = t_{i+1} - t_{i}$ where $d_{i}$ denotes the difference between two consecutive social media messages $i$ and $i+1$ that arrived in moments $t_{i}$ and $t_{i+1}$, respectively.

Using the \textit{countrytxt-usr} signal, which detect a new emergency situation and has less noise than other signals, we characterize and compare two different sub-time series within of the event. For these tasks, we extract messages one hour before earthquake and messages 10 minutes after earthquakes (Figure \ref{interarrival}). 
\begin{figure}[h]
	\centering
	\includegraphics[width=\columnwidth]{img/interarrival.png}
	\caption{Inter-arrival Time Before and During Mw 6.9 Chilean Earthquake}
	\label{interarrival}
\end{figure}

On the one hand, when an unpredicted emergency situation occurs, the urgency of messages can be represented on the first bins, where at least $60$ per cent of the messages have an inter-arrival time $d_{i} < 10$ seconds (Figure \ref{interarrival}.a). Therefore, we learn that when people share messages about a situation on a country, their location usually corresponds with said country.

On the other hand, considering a sub-time series before an earthquake, bins are spread and $28$ per cent of the messages have an inter-arrival time $d_{i} < 10$ seconds (Figure \ref{interarrival}.b). Thus, users in the same country are not simultaneously affected by the same spatio-temporal event and do not share their current location in the message.
\section{Conclusions}
Detecting an emergency situation without using a specific-domain of keywords is important because training data is hard work for researchers. We presented a proposal for detecting this event type using only locations associated with the country. These locations, unlike keywords, do not change over time and do not emerge spontaneously as new terms during an emergency situation.

In the future, we will research other emergency situations, but with local impact in the first phase after an event, e.g. terrorist attack, plane crash, etc.

\begin{thebibliography}{9}

%\bibliographystyle{chicago}
%\bibliography{bio}

\bibitem[\protect\citeauthoryear{Ashktorab, Brown, Nandi, and
	Culotta}{Ashktorab et~al.}{2014}]{ashktorab2014tweedr}
Ashktorab, Z., C.~Brown, M.~Nandi, and A.~Culotta (2014).
\newblock Tweedr: Mining twitter to inform disaster response.
\newblock {\em Proc. of ISCRAM\/}.

\bibitem[\protect\citeauthoryear{Imran, Castillo, Lucas, Meier, and
	Vieweg}{Imran et~al.}{2014}]{imranaidr2014}
Imran, M., C.~Castillo, J.~Lucas, P.~Meier, and S.~Vieweg (2014).
\newblock Aidr: Artificial intelligence for disaster response.
\newblock In {\em Proceedings of the companion publication of the 23rd
	international conference on World wide web companion}, pp.\  159--162.
International World Wide Web Conferences Steering Committee.

\bibitem[\protect\citeauthoryear{Kumar, Barbier, Abbasi, and Liu}{Kumar
	et~al.}{2011}]{kumar2011tweettracker}
Kumar, S., G.~Barbier, M.~A. Abbasi, and H.~Liu (2011).
\newblock Tweettracker: An analysis tool for humanitarian and disaster relief.
\newblock In {\em ICWSM}.

\bibitem[\protect\citeauthoryear{Olteanu, Castillo, Diaz, and Vieweg}{Olteanu
	et~al.}{2014}]{olteanu2014}
Olteanu, A., C.~Castillo, F.~Diaz, and S.~Vieweg (2014).
\newblock Crisislex: A lexicon for collecting and filtering microblogged
communications in crises.
\newblock In {\em ICWSM}.

\end{thebibliography}
\end{document}
